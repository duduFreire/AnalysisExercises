\documentclass{article}
\usepackage[utf8]{inputenc}
\usepackage[english]{babel}
\usepackage{amsmath}
\usepackage{amsthm}
\usepackage{mathtools}
\usepackage{amsfonts}

\newtheorem{theorem}{Theorem}
\newtheorem{lemma}{Lemma}

\newcounter{lemmaCounter}
\newcounter{theoremCounter}
\newenvironment{shortlemma}{\stepcounter{lemmaCounter}\textbf{Lemma~\thelemmaCounter.}}

\newenvironment{shorttheorem}{\stepcounter{theoremCounter}\textbf{Theorem~\thetheoremCounter.}}

\DeclarePairedDelimiter\abs{\lvert}{\rvert}

\newcommand{\N}{\mathbb{N}}
\newcommand{\Q}{\mathbb{Q}}
\newcommand{\R}{\mathbb{R}}
\newcommand{\ra}{\rightarrow}
\newcommand{\set}[1]{\{#1\}}
\newcommand{\exc}[2][Abbott]{\item \textbf{#1, Exercise #2.}}


\title{Real Analysis Exercises}
\author{Eduardo Freire}
\date{April 2021}

\begin{document}

\maketitle

\begin{enumerate}
	\exc[Tao]{5.4.1} For every real
	number $x$, exactly one of the following three statements is true: $(a)\, x$ is zero; $(b)\, x$ is positive; $(c)\, x$ is negative.
		      	      
	\begin{proof}
		First we show that at least one of $a$, $b$ or $c$ is true. Let $x$ be an arbitrary real number. If $x=0$ we are done. Otherwise, we need to show that either $b$ or $c$ is true. Since $x \neq 0$, it can be written as $\text{LIM}_{n \to \infty} a_n$ where $(a_n)_{n=1}^{\infty}$ is a Cauchy sequence that is bounded away from zero. Then, there is some $c > 0$ such that $|a_n| \geq c$ for all $n$. Also, there is some $N \geq 1$ such that $|a_N - a_n| \leq c/2$ for all $n \geq N$, since the sequence is Cauchy, $c/2 > 0$ and $N \geq N$. Since the sequence is bounded away from zero, none of its terms are zero. Therefore we can split the problem in two cases, $a_N > 0$ and $a_N < 0$.
			      		      	 
		Case 1 ($a_N > 0$): If we can show that $a_n \geq c/2 > 0$ we would almost be done, since we could then define a new sequence $(b_n)_{n=1}^\infty$ where $b_n := c/2$ if $n < N$ and $b_n := a_n$ if $n \geq N$, which is clearly positively bounded away from zero and equivalent to $(a_n)_{n=1}^{\infty}$. So, assume for the sake of contradiction that $a_n < c/2$. Then, $-a_n > -c/2$, therefore $a_N - a_n > a_N-c/2 \geq c/2 > 0$. But then $|a_N - a_n| = a_N - a_n \leq c/2$. Thus we have show that $c/2 < a_N-a_n \leq c/2$, a contradiction. This means that $a_n \geq c/2$ for all $n \geq N$, and we are done.
			      		      	 
		Case 2 ($a_N < 0$): Similarly to case one, we assume for the sake of contradiction that $a_n > -c/2$. Since $-a_N \geq c$, $a_n - a_N > c/2 > 0$. But then $|a_n - a_N| = a_n - a_N \leq c/2$, so we have show that $c/2 < a_n-a_N \leq c/2$, a contradiction. Therefore, for all $n \geq N$, $a_n \leq -c/2$. Now we can define a new sequence $(b_n)_{n=1}^\infty$ where $b_n := -c/2$ if $n < N$ and $b_n := a_n$ if $n \geq N$, which is clearly negatively bounded away from zero and equivalent to $(a_n)_{n=1}^{\infty}$.
			      		      	  
		Now, we show that at most one of $a$, $b$ or $c$ must be true. We do that by contradiction, in three separate cases.
			      		      	  
		Case 1 ($a$ and $b$ are true): Since x is positive, it can be written as $x = \text{LIM}_{n \to \infty} a_n$ where $(a_n)_{n=1}^{\infty}$ is positively bounded away from zero. In other words, there is some $c > 0$ such that $a_n \geq c$ for all $n \geq 1$. But since $x = 0$, $(a_n)_{n=1}^{\infty}$ is equivalent to zero, which means that there is some $N \geq 1$ such that $|a_N| = a_N \leq c/2 < c$, a contradiction.
			      		      	  
		Case 2 ($a$ and $c$ are true): Since x is negative, it can be written as $x = \text{LIM}_{n \to \infty} a_n$ where $(a_n)_{n=1}^{\infty}$ is negatively bounded away from zero. In other words, there is some $-c < 0$ such that $a_n \leq -c$ for all $n \geq 1$. But since $x = 0$, $(a_n)_{n=1}^{\infty}$ is equivalent to zero, which means that there is some $N \geq 1$ such that $|a_N| = -a_N \leq c/2$. But then $a_N \geq -c/2 > -c$, a contradiction.
			      		      	
		Case 3 ($b$ and $c$ are true): Since $x$ is both positive and negative, we have that $x = \text{LIM}_{n \to \infty} a_n = \text{LIM}_{n \to \infty} b_n$, where $(a_n)_{n=1}^{\infty}$ is positively bounded away from zero and $(b_n)_{n=1}^{\infty}$ is negatively bounded away from zero. This means that there there exists some $c_1, c_2 > 0$ such that $a_n \geq c_1$ and $-b_m \geq c_2$ for all $n,m \geq 1$. Then $a_n - b_m \geq c_1 + c_2 > 0$. However, since the sequences are equivalent, there is some $N \geq 1$ such that
		\begin{equation*}
			|a_N - b_N| = a_N - b_N \leq \frac{c_1+c_2}{2} < c_1 + c_2
			\, .
		\end{equation*}
			      		      	
		This is a contradiction, since we have already shown that $a_N - b_N \geq c_1 + c_2$ 
			      		      	 
	\end{proof}
		      	      
	\exc[Tao]{5.5.2} Let $E$ be a non-empty subset of $\mathbb{R}$, let $n \geq 1$ be an integer, and
	let $L<K$ be integers. Suppose that $K/n$ is an upper bound for $E$, but that
	$L/n$ is not an upper bound for $E$. Without using the Least Upper Bound Theorem, show that there exists an integer $L < m \leq K$ such that $m/n$ is an upper bound for $E$, but that $(m - 1)/n$ is not an upper bound for $E$.
		      	      
	\begin{proof}
		We will say a real number $w$ is U.B whenever $w$ is an upper bound for $E$.
			      		      	
			      		      	
		Suppose for sake of contradiction that there is no integer $L < m \leq K$ such that $m/n$ is U.B but that $(m - 1)/n$ is not U.B. This implies that if $m/n$ is U.B, $(m - 1)/n$ must also be U.B (as long as $L < m \leq K$).
		Let $P(t)$ be the statement "$L < K-t \implies$ Both $(K-t)/n$ and $(K-t-1)/n$ are U.B". We will prove $P(t)$ holds for all natural $t$ by induction. First, we need to show that $L < K \implies$ Both $K/n$ and $(K-1)/n$ are U.B. $K/n$ is U.B by assumption, and since $L < K \leq K$, $(K - 1)/n$ also has to be an U.B, again by assumption. Now assume $P(t)$. We need to show that $L < K-t-1 \implies$ Both $(K-t-1)/n$ and $(K-t-2)/n$ are U.B. If $ L \geq K-t-1$, $P(t+1)$ is vacuously true, so we assume $K \geq K-t-1 > L$. Notice that $K-t-1 > L \implies K-t > L$. By the induction hypothesis, $(K-t-1)/n$ is U.B, but this also means that $(K-t-2)/n$ is U.B, as we wanted to show.
			      		      	
		Now, since $K>L$ we have that $K \geq L+1$, which means that $K = L+1+c$ for some natural number $c$. Then $P(c)$ holds and also $L < K- c = L + 1 \leq K$. Therefore $(K-c-1)/n = L/n$ must be U.B, a contradiction.
	\end{proof}
		      	      
	\exc[Tao]{5.4.5} Given any two real numbers $x<y$, we can find a rational number $q$ such that $x<q<y$.
	\begin{proof}
		By the Archimedean Property, there is a positive integer $b$ such that $(y-x)b > 1 > 0$. Since $x-y$ is positive and their product with $b$ is also positive, it follows that $b > 0$. By Exercise 5.4.3, there is an integer $a-1$ such that $a-1 \leq bx < a$. By the definition of $b$, we have that $bx < by - 1$. Then, $a-1 \leq bx < by-1 \implies a < by$. Now we have $bx < a < by$. Since $b > 0$, we can divide this inequality by $b$ resulting in $x < a/b < y$. We can define $q := a/b$ and since $a$ and $b$ are integers (with $b \neq$ 0) we are done.
	\end{proof}
		      	      
	\exc[Tao]{5.4.8} Let $(a_n)_{n=0}^\infty$ be a Cauchy sequence of rationals, and let $x$ be
	a real number. Show that if $a_n \leq x$ for all $n \geq 1$, then $\text{LIM}_{n \to \infty} a_n \leq x$
	Similarly, show that if $an \geq x$ for all $n \geq 1$, then $n \geq 1$, then $\text{LIM}_{n \to \infty} a_n \geq x$.
		      	      
	\begin{proof}
		Assume $a_n \leq x$ for all $n \geq 1$. For sake of contradiction, assume that $\text{LIM}_{n \to \infty} a_n > x$. In that case, we can find a rational $q$ such that $\text{LIM}_{n \to \infty} a_n > q > x \geq a_n$. Since $q > a_n$ for all $n \geq 1$, Corollary 5.4.10 asserts that $q \geq \text{LIM}_{n \to \infty} a_n$. Now, we have that $\text{LIM}_{n \to \infty} a_n > q \geq \text{LIM}_{n \to \infty} a_n$, a contradiction. A very similar proof follows when $x \leq a_n$.
	\end{proof}
		      	      
	\exc{1.3.2}
	\begin{enumerate}
		\item A real number $s$ is the \textit{greatest lower bound} for a set $A \subseteq \mathbb{R}$ if and only if it meets the following criteria:
		      \begin{enumerate}
		      	\item $s$ is a lower bound for $A$;
		      	\item If $b$ is a lower bound for $A$, then $s \geq b$.
		      \end{enumerate}
		\item We want to show that if $s \in \mathbb{R}$ is a lower bound for a set $A \subseteq \mathbb{R}$, then $s = \inf A$ if and only if for every $\epsilon > 0$, there exists an element $a \in A$ such that $s + \epsilon > a$.
		      \begin{proof}
		      	For the forward direction, assume $s = \inf A$. Since $s + \epsilon > s$, it is not a lower bound for $A$. Therefore, there must be some $a \in A$ such that $s + \epsilon > a$.
		      		      	      		      	      	    
		      	Conversely, assume $s$ is a lower bound for $A$, with the property that for every $\epsilon > 0$ there is some $a \in A$ such that $ s + \epsilon > a$. Let $b$ be a lower bound for $A$. Assume for sake of contradiction that $b > s$. then we can choose $\epsilon = b - s > 0$ and we have that $b = s + \epsilon > a$, which means that $b$ is not a lower bound for $A$, a contradiction. Therefore $b \leq s$, thus $s = \inf A$.
		      \end{proof}
	\end{enumerate}
		      	      
	\pagebreak
	\exc{1.3.3}
		      	      
	\begin{enumerate}
		\item First, we show that $s := \sup B \geq m$ for any $m$ that is a lower bound for $A$. Since $m$ is a lower bound for $A$, $m \in B$. But, since $s$ is an upper bound for $B$, $s \geq m$. Now we need to show that given any $a \in A$, $s \leq a$. Let $b \in B$ be arbitrary. By Lemma 1.3.7, for every choice of $\epsilon > 0$, $s-\epsilon < b$. Since $b$ is a lower bound for $A$, we have that $s-\epsilon < b \leq a$, therefore $s \leq a$. 
		      	      	      	      	          
		\item The previous item gives a general process for finding the greatest lower bound of any nonempty $A \subseteq \mathbb{R}$ that is bounded below. Namely, construct a set $B = \{b \in \mathbb{R} : b \text{ is a lower bound for }A\}$. Then ,since $A$ is bounded below, $B$ is nonempty. $B$ is also bounded above, since any $a \in A$, is an upper bound for $B$. This follows because if $b \in B$, then $b$ is a lower bound for $A$, therefore $a \geq b$. Then, the Axiom of Completeness guarantees that $s := \sup B$ exists, and we have already shown that this means $s = \inf A$. This means that the existence of the greatest lower bound is a corollary of Completeness, hence needs not be asserted by it.
		      	      	      	      	          
		\item Let $A \subseteq \mathbb{R}$ be nonempty and bounded below. Then, define the set $-A = \{-x: x \in A\}$. $-A$ is clearly nonempty. Also, let $m$ be a lower bound for $A$ and pick an $x \in -A$. Then, $x = -a$ for some $a \in A$. But then $a \geq m \implies x=-a \leq -m$. This shows that $-m$ is an upper bound for $-A$, therefore $-A$ is both nonempty and bounded above. Then, by the Axiom of Completeness, there exists an $s := \sup -A$. We want to show that $-s = \inf A$. We have that $s \geq x = -a$. Then, $-s \leq a$ thus $-s$ is a lower bound for $A$. Now, let $b \in \mathbb{R}$ be a lower bound for $A$. Then, $-b$ is an upper bound of $-A$, therefore $-b \geq \sup -A = s$. This means that $b \leq -s$, and $-s$ is indeed the greatest lower bound of $A$.
	\end{enumerate}
		      	      
	\exc{1.3.4}
		      	      
	Since $B \subseteq A$, it follows that if $b \in B$, then $b \in A$. Lets pick one such $b$. Then we must have $b \leq \sup B$, but also $b \leq \sup A$
	, since $b$ is an element of $A$. This means that $\sup A$ must be an upper bound of $B$, therefore $\sup B \leq \sup A$.
		      	      
	\exc{1.3.5}
		      	      
	\begin{enumerate}
		\item Let $x$ be an element of $c + A$. Then, there is some $a \in A$ such that $x = c + a$. Then, since $\sup A \geq a$ we have $c + \sup A \geq c + a = x$, therefore $c + \sup A$ is an upper bound for $c + A$. Now, let $b$ be an upper bound of $c + A$. Then, $b \geq x = c+a$, which means that $b - c \geq a$, so $b-c$ is an upper bound for $A$. This means that $b-c \geq \sup A$, thus $b \geq c + \sup A$, and $\sup (c+A) = c + \sup A$, as desired.
		      	      	      	      	          
		\item If $c = 0$, then the only element of $c A$ is $0 = \sup c A = c \sup A$. Then, we can assume $c > 0$ and proceed very similarly to the previous item. Let $x$ be an element of $c A$. The, there is some $a \in A$ such that $x = c a$. Then, since $\sup A \geq a$ we have $c \sup A \geq c a = x$, therefore $c \sup A$ is an upper bound for $c A$. Now, let $b$ be an upper bound of $c A$. The, $b \geq x = c A$, which means that $b/c \geq a$, so $b/c$ is an upper bound of $A$. This means that $b/c \geq \sup A$, thus $b \geq c \sup A$, and $\sup (c A) = c \sup A$, as desired.
		      	      	      	      	          
		\item If $c < 0$, then $\sup(c A) = c\inf(A)$.
	\end{enumerate}
		      	      
	\exc{1.3.6}
		      	      
	\begin{enumerate}
		\item $3$, $1$.
		\item $1$, $0$.
		\item $1/3$, $1/2$
		\item $9$, $1/9$
	\end{enumerate}
		      	      
	\exc{1.3.7}
		      	      
	Let $b$ also be an upper bound for $A$. Then $b \geq a$, therefore $a = \sup A$
		      	      
	\exc{1.3.8}
	Let $\epsilon = \sup B - \sup A$. Since $\sup B > \sup A$, $\epsilon$ is positive, which means there is an element $b \in B$ such that $\sup A = \sup B - \epsilon < b$. Now, let $a$ be an element of $A$. Then, $a \leq \sup A < b$, therefore $b$ is an upper bound for $A$.
		      	      
	\exc{1.3.9}
	\begin{enumerate}
		\item True.
		\item False. $L = 2$ and $A = (0, 2)$.
		\item False. $A = (0, 1)$, $B = (1, 2).$
		\item True.
		\item False. $A = B = (0, 1).$
	\end{enumerate}
		      	       
	\exc{1.4.1}
		      	       
	If $b > 0$, then $a < 0 < b$ hence we can set $r := 0$ since $0 \in \mathbb{Q}$. Now, assume $b \leq 0$. Then, $0 \leq -b < -a$ so we can choose an $r \in \mathbb{Q}$ such that $-b < r < -a$, therefore $a < -r < b$ and we are done ($-r$ is also rational). 
		      	       
	\exc{1.4.2}
	Since $a,b \in \mathbb{Q}$, there are integers $r_1, r_2, q_1,q_2$ such that $a = r_1/q_1$ and $b = r_2/q_2$.
		      	        
	\begin{enumerate}
		\item \begin{equation*}
		      a + b = \frac{r_1}{q_1} + \frac{r_2}{q_2} = \frac{r_1 q_2 + r_2 q_1}{q_1 q_2} 
		\end{equation*}
			      		      	      
		\begin{equation*}
			ab = \frac{r_1 r_2}{q_1 q_2}
		\end{equation*}
			      		      	      
		Since $r_1 q_2 + r_2 q_1$ and $q_1 q_2 \neq 0$ are integers, $a+b$ and $ab$ are rational numbers.
			      		      	      
		\item Assume $a + t \in \mathbb{Q}$. Then, 
		      \begin{equation*}
		      	a + t = \frac{r_1}{q_1} + t = \frac{r_3}{q_3}
		      \end{equation*}
		      	      	      	      	            
		      for some $r_3, q_3 \in \mathbb{Z}$ with $q_3 \neq 0$. But then,
		      \begin{equation*}
		      	t = \frac{r_3}{q_3} +  \frac{-r_1}{q_1}
		      \end{equation*}
		      	      	      	      	            
		      which is a sum of rational numbers, therefore also rational, a contradiction. Since $\mathbb{R}$ is closed under addition, $t$ must be irrational.
		      	      	      	      	            
		\item $\mathbb{I}$ is not closed under addition or multiplication. If $t$ is an irrational number and $q$ is rational, then $s := q-t$ is also irrational. However, $t + s = q$ is a rational number, therefore two irrationals can sum to a rational. Also, if we instead set $s := q/t$, then $t s = q$, so irrationals are also not closed under multiplication.
	\end{enumerate}
		      	        
	\exc{1.4.3}
	Since $a < b$, $a - \sqrt{2} < b - \sqrt{2}$, and we can find a rational number $q$ such that $a - \sqrt{2} < q < b - \sqrt{2}$. We then have $a < q + \sqrt{2} < b$. Since $q \in Q$, $q + \sqrt{2}$ must be irrational.
		      	      
	\exc{1.4.4}
	Let $A := \{1/n : n \in \mathbb{N}\}$.
	Assume there is some $n \in \mathbb{N}$ such that $1/n < 0$. Multiplying by $n$ makes the contradiction clear, so $0$ is a lower bound for $A$. Now, let $b$  also be a lower bound for $A$ and assume $b > 0$. Then, we can use the Archimedean Property of $\mathbb{R}$ to find some $n \in \mathbb{N}$ such that $1/n < b$. But this contradicts the fact that $b$ is a lower bound for $A$, therefore we must have $b \leq 0$, which proves $0 = \inf A$.
		      	        
	\exc{1.4.5} 
	Assume that $\bigcap_{n=1}^\infty (0, 1/n) \neq \emptyset$. Then, there must be some $x \in \mathbb{R}$ such that for all $n \in \mathbb{N}$, $0 < x < 1/n$. But then we can use the Archimedean Property to find a natural $m$ such that $x > 1/m$, which is a contradiction. Therefore $\bigcap_{n=1}^\infty (0, 1/n) = \emptyset$.
		      	        
	\exc{1.4.6}
	The following has already been shown:
	\begin{equation*}
		(\alpha-\frac{1}{n})^2 > \alpha^2 - \frac{2\alpha}{n} \, .
	\end{equation*}
	Now, choose $n_0 \in \mathbb{N}$ such that
	\begin{equation*}
		\frac{1}{n_0} < \frac{\alpha^2-2}{2\alpha} \, .
	\end{equation*}
	It follows that 
	\begin{equation*}
		(\alpha-\frac{1}{n_0})^2 > \alpha^2-\frac{2\alpha}{n_0} > 2 > t^2
	\end{equation*}
	for any $t \in T$. Since $\alpha = \sup T$, it follows that there is some $r \ in T$ such that $\alpha - 1/n_0 < r$. But, since $(\alpha-1/n_0)^2 > r^2$ and $\alpha - 1/n_0 > 0$, we have $\alpha - 1/n_0 > r$, a contradiction.
		      	        
	\exc{1.4.7}
	Let $C_1 = \{n \in \mathbb{N} : f(n) \in A\}$ and $C_{k+1} = C_k \backslash \{n_k\}$ for all $k \in \mathbb{N}$, where $n_k := \min C_k$. Then, let $g : \mathbb{N} \rightarrow A$ be defined as $g(k) = f(n_k)$. First we prove a couple of useful lemmas.
		      	        
	Lemma 1: For all natural $k$, $n_{k+1} > n_k$. Also, $a \neq b \implies n_a \neq n_b$. \begin{proof}
	In the first part, since $n_{k+1} \in C_{k+1}$, we have that $n_{k+1} \in \{n \in C_k : n \neq n_k\}$, therefore $n_{k+1} \in C_k$, so $n_{k+1} > n_k$, since $n_k$ is the minimum of $C_k$ and $n_{k+1} \neq n_k$.
		      	        
	Now, assume $a,b \in \mathbb{N}$ and $a \neq b$. Without loss of generality, also assume that $a > b$. Then $n_a > n_b$, therefore $n_a \neq n_b$.
	\end{proof}
			  
	Lemma 2: For every $L \in \mathbb{N}$ such that $f(L) \in A$, there is some $k \in \mathbb{N}$ such that $n_k = L$.
	\begin{proof}
		Let $L$ be a natural number such that $f(L) \in A$. Then, $L \in C_1$. Now, assume for sake of contradiction that there is no natural $k$ such that $n_k = L$. This means that for all $k \in \mathbb{N}$, we can find a $c_k \in C_k$ such that $L > c_k$, since $L$ cannot be the minimum of $C_k$. Now, pick a $c_L \in C_L$ such that $L > c_L$. Then, $c_L$ must be grater than or equal to the minimum of $C_L$, namely $n_L$. Now, we can use Lemma 1 to see that $L > n_L > n_{L-1} > \dots > n_1 > 0$. This shows that there are $L$ natural numbers strictly between $0$ and $L$, which is a contradiction.
	\end{proof}
			  
	Now, we show that $g$ is onto. Let $a$ be an element of $A$. Since $f$ is onto and $A \subseteq B$, there must be a natural number $L$ such that $f(L) = a$. Then, by Lemma 2, we have a $k \in \mathbb{N}$ such that $n_k = L$. Now, we have that $f(L) = f(n_k) = g(k) = a$, as we wanted to show.
			  
	Next, we show that $g$ is one-to-one. Assume that $g(k_1) = g(k_2)$, for $k_1, k_2 \in \mathbb{N}$. This means that $f(n_{k_1}) = f(n_{k_2})$, and since $f$ is one-to-one, it must be that $n_{k_1} = n_{k_2}$. By Lemma 1, this can only happen when $k_1 = k_2$, and we are done.
			  
	\exc{1.4.8}
	\begin{enumerate}
		\item $B_2$ is a subset of $A_1$, therefore it is countable or finite. First, assume it is countable. Then, there must be two functions $f : \mathbb{N} \rightarrow A_1$ and $g : \mathbb{N} \rightarrow B_2$ which are 1-1 and onto. Then, we can define another function $h : \mathbb{N} \rightarrow A_1 \cup B_2$ as following:
		      \begin{equation*}
		      	h(n) = \begin{cases}
		      	f(\frac{n-1}{2}) & \text{n is odd} \\
		      	g(\frac{n}{2}), & \text{n is even}
		      	\end{cases}
		      \end{equation*}
		      	      	      	      	        
		      To show that $h$ is 1-1, assume that $a \neq b$ for natural numbers $a$ and $b$. If they are both odd, then $h(a) = f((a-1)/2)$ and $h(b) = f((b-1)/2)$. Since $(a-1)/2 \neq (b-1)/2$ and $f$ is 1-1, we must have $h(a) \neq h(b)$.
		      A very similar argument follows if $a$ and $b$ are both even. The final case is $a$ is odd and $b$ is even. Then, $h(a) = f((n-1)/2)$ and $h(b) = g(b/2)$. Since $f((n-1)/2) \in A_1$ and $g(b/2) \in B_2$, it must be the case that $h(a) \neq h(b)$, since $g(b/2) \notin A_1$. Now, we need to show that $h$ is onto. Let $t \in A_1 \cup B_2$. We must find an $n \in \mathbb{N}$ such that $h(n) = t$. First, assume $t \in A_1$. Since $f$ is onto, there is a $k \in \mathbb{N}$ such that $f(k) = t$. Setting $n := 2k + 1$, we have that $h(n)=f(k)=t$. Next, assume $t \in B_2$. Since $g$ is onto, there is a $k \in B_2$ such that $g(k) = t$. Setting $n := 2k$, we have that $h(n)=g(k)=t$.
		      This shows that $A_1 \cup B_2 \sim \mathbb{N}$ is countable whenever $B_2$ is countable.
		      	      	      	      	        
		      Next, we informally discuss why the theorem holds if $B_2$ is finite. In this case, we can find a bijection $g: \{1, 2, 3, \dots, m\} \rightarrow B_2$, where $m$ is the cardinality of $B_2$. Now, we define a new function $h: \mathbb{N} \rightarrow A_1 \cup B_2$ as following: 
		      \begin{equation*}
		      	h(n) = \begin{cases}
		      	g(n) & n \leq m \\
		      	f(n-m) & n > m
		      	\end{cases}
		      \end{equation*}
		      where $f : \mathbb{N} \rightarrow A_1$ is a bijection. Now pick two numbers $a, b \in \mathbb{N}$ and assume $a \neq b$. WLOG, we can make $a > b$. If $a > b > m$, then $h(a) = f(a-m)$ and $f(b-m) = h(b)$. Since $f$ is 1-1, $h(a) \neq h(b)$. Next, if $b \leq m$,  then $h(b) = g(b)$. In the case where $a \leq m$, we also have $h(a) = g(a) \neq g(b)$. Otherwise, $h(a) = f(a-m) \in A_1$, and, since $h(b) \notin A_1$, $h(a) \neq h(b)$, therefore $h$ is 1-1.
		      	      	      	      	        
		      Now, we need to show that $h$ is onto. Let $t \in A_1 \cup B_2$. We must find an $n \in \mathbb{N}$ such that $h(n) = t$. First, assume $t \in A_1$. Since $f$ is onto, there is a $k \in \mathbb{N}$ such that $f(k) = t$. Setting $n := m + k$, we have that $h(n) = f(k) = t$. Next, assume $t \in B_2$. Since $g$ is onto, there is a $k \in B_2$ such that $g(k) = t$. Setting $n := k$, we have that $h(n)=g(k)=t$, since $k \in B_2 \implies k \leq m$.
		      	      	      	      	        
		      The more general statement in (i) follows easily by applying induction to the statement just proved.
		      	      	      	      	        
		\item We can use induction to show that $\bigcup_{n=1}^m A_n$ is countable for any particular $m \in \mathbb{N}$, which only consists of a finite ($m$) number of unions, not infinite.
		      	      	      	      	        
		\item Each one of the columns has a countable number of elements, and there are countably many columns. By matching each $a_m \in A_n$ with the $n$th column, and $m$th row, we have created a bijection between the unions of all the $A_n$ and the natural numbers.
	\end{enumerate}
		      	          
	\exc{1.4.9}
		      	          
	\begin{enumerate}
		\item Let $f : A \rightarrow B$ be a bijection. Since for every $b \in B$ there is a unique $a \in A$ such that $f(a) = b$, we can define a new function $g: B \rightarrow A$ where $g(b) = g(f(a)) = a$. To show that $g$ is 1-1, let $g(b_1) = g(b_2)$, where $b_1,b_2 \in B$. Then, we can find $a_1, a_2 \in A$ such that $f(a_1) = b_1$ and $f(a_2) = b_2$. Then, $g(f(a_1)) = g(f(a_2))$ and by the definition of $g$ this means that $f(a_1) = f(a_2)$, therefore $a_1 = a_2$. Now, $b_1 = f(a_1) = f(a_2) = b_2$, so $g$ is 1-1. Next, let $a \in A$. We must find a $b \in B$ such that $g(b) = a$. All we have to do is set $b := f(a)$, and we are done. Since $g$ is a bijection between $B$ and $A$, it follows that $B \sim A$.
		      	      	      	      	              
		\item Let $f : A \rightarrow B$ and $g : B \rightarrow C$ be bijections and $h : A \rightarrow C$ be a function such that for every $a \in A$, $h(a) = g(f(a))$. This can be done since $f(a) \in B$ and $g(f(a)) \in C$. To show that $h$ is 1-1, let $h(a_1) = h(a_2)$. This means that $g(f(a_1)) = g(f(a_2))$, then $f(a_1) = f(a_2)$ and finally $a_1 = a_2$. Now, pick a $c \in C$.  Since $g$ is onto, there is a $b \in B$ such that $g(b) = c$, and since $f$ is onto, there is an $a \in A$ such that $f(a)=b$. Then $g(f(a)) = h(a) = c$, which shows $h$ is onto. Since $h$ is also 1-1, it follows that $A \sim C$.
	\end{enumerate}
		      	          
	\exc{1.4.10}
	Let $S_n := \{S \subseteq \mathbb{N} : \text{The cardinality of } S = n\}$ for every $n \in \mathbb{N}$. Then, the set of all finite subsets of $\mathbb{N}$ is $U = \bigcup_{n=1}^\infty S_n$. If we can show that each $S_n$ is countable, Theorem 1.4.13 guarantees $U$ is also countable.
		      	          
	Define $T_{1,m} = S_1$ and $T_{n+1,m} = \{\{m\} \cup s : s \in S_n, m \notin s\}$ for all $n, m \in \mathbb{N}$. We claim that \begin{equation*}
	S_n = \bigcup_{m=1}^{\infty} T_{n,m} \, .
	\end{equation*}
			    
	This is clearly true when $n = 1$, so we now show that the equality holds for all $n > 1$. To see that, let $x \in \bigcup_{m=1}^{\infty} T_{n,m}$. Then $x \in T_{n, m}$ for some $m \in \mathbb{N}$. This means that $x = \{m\} \cup s$, where $s \in S_{n-1}$, thus $x \subseteq \mathbb{N}$. Since $m \notin s$, the cardinality of $x$ is $n$, so $x \in S_n$. Now, we must show that $x \in S_n \implies x \in \bigcup_{m=1}^{\infty} T_{n,m}$, and we will call this statement $P(n)$. Assume $x \in S_n$. If $n = 1$, we have already seen that the equality in question holds, so $P(1)$ is true. Now assume $P(n)$. Also, let $y \in S_{n+1}$. Then, the cardinality of $y$ is $n+1$. Next, we can use the fact that $y \subseteq \mathbb{N}$ to see that $y = \{m\} \cup s$ for some $s \in S_n$ and some natural $m \notin s$. But this means that $y \in T_{n+1, m}$, so $P(n+1)$ holds.
			    
	Now, lets show by induction that $T_n$ is countable. $T_{1,m} = S_1$ is easily seen to be countable by defining a function $v : \mathbb{N} \rightarrow T_{1,m} $ such that $v(n) = \{n\}$. Now, assume $T_{n,m}$ is countable and define the set $A_m := \{a \in \mathbb{N} : m \notin f(a)\}$. By Theorem 1.4.13, $S_n$ is also countable, therefore there is a bijection $f : \mathbb{N} \rightarrow S_n$. Define a function $g : A\rightarrow T_{n+1, m}$ such that $g(a) = \{m\} \cup f(a)$, and let $a_1, a_2 \in A$ with $a_1 \neq a_2$. Then, $f(a_1) \neq f(a_2)$, and since $m \notin f(a_1), f(a_2)$, we have that $\{m\} \cup f(a_1) \neq \{m\} \cup f(a_1)$, thus $g(a_1) \neq g(a_2)$ so $g$ is 1-1. Now, let $t \in T_{n+1, m}$. Then, $t = \{m\} \cup s$ where $m \notin s$. Since $f$ is onto, there is some $n \in \mathbb{N}$ such that $f(n) = s$. Then, $g(n) = \{m\} \cup s = t$, so $g$ is onto. This shows that $A \sim T_{n+1, m}$. Since $A \subseteq \mathbb{N}$ and $A$ is not finite, it must be countable, therefore every $T_{n, m}$ is also countable, and Theorem 1.4.13 can be used to see that this results in every $S_n$ being countable, as we wanted to show.
			    
	\exc{1.4.11}
	\begin{enumerate}
		\item $f : (0,1) \rightarrow S$, $f(x) = (x, 1/2)$.
		      	      	      	      	              
		\item For every $x \in \mathbb{R}$ if there is a decimal expansion of $x$ that ends in a tail of nines, we can instead choose one that ends in a tail of zeros, and we will call this the unique expansion of $x$ (when $x$ does not end in a tail of nines the expansion is already unique). Then, let $(x_1, x_2) \in S$. We can expand $x_1$ and $x_2$ uniquely as follows:
		      	      	      	      	              
		      \begin{gather}
		      	\nonumber x_1 = 0.d_1 d_2 d_3 \dots \\
		      	\nonumber x_2 = 0.e_1 e_2 e_3 \dots
		      \end{gather}
		      Then, define the function $f : S \rightarrow (0, 1)$ as following: 
		      \begin{equation*}
		      	f((x_1, x_2)) = 0.d_1e_1d_2e_2 \dots
		      \end{equation*}
		      	      	      	      	             
		      $f$ is 1-1, but not onto. Consider for example $x = 0.898989\dots$. Notice that every other digit is a $9$, so in order for $f$ to map some ordered pair to $x$, one of the elements of the pair would have to be $0.999\dots = 1$, which is not in the domain of $f$.
	\end{enumerate}
		      	          
	\exc{1.5.11} (Switched to second edition here)
		      	          
	\begin{enumerate}
		\item Since $g$ maps $B'$ onto $A'$, for every $a \in A'$ there is a $b \in B'$ such that $g(b) = a$. Since $g$ is also 1-1, this $b$ is unique. Then, we can define a function $g^{-1} : A' \rightarrow B'$ such that $g^{-1}(a) = b$. To show that $g^{-1}$ is onto, let $y \in B'$ be arbitrary. Then $g(y) = x$ for some $x \in A'$, which by the definition of $g^{-1}$ means that $g^{-1}(x) = y$ so $g^{-1}$ is onto. Next, we show that $g{-1}$ is 1-1 by letting $g^{-1}(x_1) = g^{-1}(x_2)$ for some $x_1, x_2 \in A'$. Then, we can find $y_1, y_2 \in B'$ such that $g(y_1) = x_1$ and $g(y_2) = x_2$. Then, $g^{-1}(g(y_1)) = g^{-1}(g(y_2))$, in other words, $g(y_1) = g(y_2)$, which means $y_1 = y_2$ since $g$ is 1-1. Then, $g(y_1) = x_1 = g(y_2) = x_2$, and $g^{-1}$ is 1-1 and onto.
		      	      	      	      	              
		      Now, let $h : X \rightarrow Y$ be such that 
		      \begin{equation*}
		      	h(x) = \begin{cases}
		      	f(x) & x \in A \\ 
		      	g^{-1}(x) & x \in A'
		      	\end{cases}
		      \end{equation*}
		      	      	      	      	              
		      for every $x \in X$. Now, assume $a \neq b$ for $a, b \in X$. If $a$ and $b$ are elements of $A$, then $h(a) \neq h(b)$, since $f$ is 1-1. Also, if $a,b \in A'$, then $h(a) \neq h(b)$ since $g^{-1}$ is 1-1. The last case is $a \in A$ and $b \in A'$, then $h(a)=f(a) \in B$, and $h(b)=g^{-1}(b) \in B'$, and we can use the fact that $A'$ and $B'$ are disjoint to see that $h(a) \neq h(b)$, so $h$ is 1-1. Now, let $y \in Y$. If $y \in B$, then there is some $a \in A$ such that $f(a)=h(a)=y$, since $f$ maps $A$ onto $B$. Also, if $y \in B'$, then there is some $a' \in A'$ such that $g^{-1}(a') = y$, since $g^{-1}$ is onto.
	\end{enumerate}
		      	          
	\exc{1.6.1}
	The function $(1-2x)/((2x-1)^2-1)$ maps $(0, 1)$ to $\mathbb{R}$ both 1-1 and onto, therefore $\mathbb{R} \sim (0, 1)$, and since $\sim$ is an equivalence relation, $\mathbb{R}$ is uncountable $\iff (0, 1)$ is uncountable. 
		      	          
	\exc{1.6.2}
	\begin{enumerate}
		\item If $a_{11} = 2$, then $b_1 = 3$, and if $a_{11} \neq 2$, $b_1 = 2$. In both cases, $a_{11} \neq b_1$. Since $f(1) = .a_{11} a_{12} \dots$, $x$ and $f(1)$ differ in at least one decimal place, therefore they are not equal.
		      	      	      	      	              
		\item If $a_{nn} = 2$, then $b_n = 3$, and if $a_{nn} \neq 2$, $b_n = 2$. In both cases, $a_{nn} \neq b_1$. Since $f(n) = .a_{n 1} \dots a_{nn} a_{n n+1} \dots$, $x$ and $f(n)$ differ in at least one decimal place , therefore they are not equal.
		      	      	      	      	              
		\item We assumed that every real number is included in the list, therefore there is some $n \in \mathbb{N}$ such that $x = f(n)$. However, we have also shown that this cannot be the case, which is a contradiction. Therefore, our assumption that $(0, 1)$ must be false, and $(0, 1)$ is uncountable.
	\end{enumerate}
		      	          
	\exc{1.6.3}
		      	          
	\begin{enumerate}
		\item We cannot apply the same argument to $\mathbb{Q}$ because even though every rational number has a decimal expansion, it is not true that every decimal expansion corresponds to a rational number. Therefore, the number $x = .b_1b_2 \dots$ created is only guaranteed to be a real number, so we cannot use the fact that $x$ is not in the list to get a contradiction. Instead, this argument shows that the number $x$ must be irrational.
		      	      	      	      	              
		\item We used the fact that if $x, y \in \mathbb{R}$ and the $n$th digit of $x$ is not equal to the $n$th digit of $y$, then $x \neq y$. However, $0.499\dots = 0.5$, and their first digits (after the decimal point) are different. Fortunately, this only happens when one of $x$ has a decimal expansion that terminates, and $y$ can be written with repeating nines (or vice-versa), and this is never the case with the real number $x$ that we constructed, since its only digits are $2$ and $3$. 
	\end{enumerate}
		      	          
	\exc{1.6.4}
	Assume that $S$ is countable. Then, there is a function $f : \mathbb{N} \rightarrow S$ which is 1-1 and onto. Now, let $(a_n)$ be a sequence such that \begin{equation*}
	a_n = \begin{cases}
	0 & f(n)_n = 1 \\ 
	1 & f(n)_n = 0
	\end{cases}
	\end{equation*}
	where $f(n)_n$ represents the $n$th entry in the sequence $f(n)$. Since $a_n$ is a sequence of only zeros and ones, $(a_n) \in S$. Since $f$ is onto, this means that there is some $k \in \mathbb{N}$ such that $f(k) = (a_n)$. However, we know that $f(k)_k \neq a_k$, therefore $f(k) \neq (a_n)$, a contradiction. This means that $S$ is not countable. Since $S$ is also infinite, $S$ is uncountable.
			    
	\exc{1.6.5}
		      	          
	\begin{enumerate}
		\item $P(A) = \{\emptyset, \{a\}, \{b\}, \{c\}, \{a, b\}, \{a, c\}, \{b, c\}, \{a, b, c\}\}$.
		      	      	      	      	              
		\item If $A$ has $1$ element $a$, then $P(A) = \{\emptyset, \{a\}\}$ has $2^1 = 2$ elements. Now assume that if $A$ has $n$ elements $P(A)$ has $2^n$ elements. Let $B$ have $n + 1$ elements, and $b \in B$. The set $B' = B \backslash \{b\}$, has cardinality $n$, so $P(B')$ has $2^n$ elements. But every element of $P(B)$ is either an element of $P(B')$ or the union of one of the elements of $B'$ with $b$. Therefore, $P(B)$ has $2^n + 2^n = 2^{n+1}$ elements.
	\end{enumerate}
		      	          
	\exc{1.6.6}
		      	          
	\begin{enumerate}
		\item \begin{equation*}
		      f(x) = \begin{cases}
		      \emptyset & x = a \\
		      \{a\} & x = b \\
		      \{b\} & x = c
		\end{cases}
		\end{equation*}
		\begin{equation*}
			g(x) = \begin{cases}
			\{a\} & x = a \\
			\{b\} & x = b \\
			\{c\} & x = c
			\end{cases}
		\end{equation*}
			      		      	        
		\item \begin{equation*}
		      g(x) = \begin{cases}
		      \{1\} & x = 1 \\
		      \{2\} & x = 2 \\
		      \{3\} & x = 3 \\ 
		      \{4\} & x = 4
		\end{cases}
		\end{equation*}
			      		      	        
		\item Since there are more elements in $P(C)$ than $C$, a mapping from $C \rightarrow P(C)$ always "runs out of" elements from $C$ before mapping all to all of the elements in $P(C)$.
	\end{enumerate}
		      	          
	\exc{1.6.8}
		      	          
	\begin{enumerate}
		\item By the definition of $B$, $a'$ is some element of $A$ such that $a' \notin f(a') = B$. Since we assumed $a' \in B$, this is a contradiction.
		      	      	      	      	              
		\item Since $a' \notin B$ and $a' \in A$, it must be the case that $a' \in f(a') = B$, a contradiction.
	\end{enumerate}
		      	          
	\exc{1.6.9}
	Let $A \in P(\mathbb{N})$ be an arbitrary subset of the naturals. Then, define the function $f : P(\mathbb{N}) \rightarrow S$ such that
	\begin{equation*}
		f(A)_n = \begin{cases}
		0 & n \notin A \\
		1 & n \in A
		\end{cases}
	\end{equation*}
	where $S$ is the set of all sequences of $0'$s and $1'$s discussed in Exercise 1.6.4, and $f(A)_n$ stands for the $n$th term of the sequence $f(A)$. Since $S$ is uncountable, if we can show that $f$ is 1-1 and onto, then $P(\mathbb{N}) \sim S$, which is uncountable. Assume that $f(X) = f(Y)$ for some $X, Y \subseteq \mathbb{N}$. This means that for all $n \in \mathbb{N}$ $f(X)_n = f(Y)_n$. Now, pick an arbitrary $n \in X$. Then, $f(X)_n = 1 = f(Y)_n$, which means $n$ must also be an element of $Y$. A very similar argument follows if you first pick an $n \in Y$. This means that $n \in X \iff n \in Y$, so $X = Y$ and $f$ is 1-1. Now, let $s \in S$ be arbitrary. To show that $f$ is onto, we must find some $A \subseteq \mathbb{N}$ such that $f(A) = s$. To do that let $A = \{a \in \mathbb{N}:s_a = 1\}$. Then $f(A)_n = 1$ means that $n \in A$, which only happens if $s_n$ is also equal to $1$, so $f(A_n) = s_n$ in this case. Finally, if $f(A)_n = 0$, then $n \notin A$, so $s_n \neq 1$, which can only happen if $s_n = 0 = f(A)_n$, therefore $f(A) = s$ and $f$ is onto. 
		      	          
	We have shown that $P(\mathbb{N}) \sim S$, but our goal was to show that $P(\mathbb{N}) \sim \mathbb{R}$. We do this by showing that $S \sim (0, 1)$. Since $(0, 1) \sim \mathbb{R}$ and $\sim$ is an equivalence relation this automatically gives our wanted result. To do that, let $x \in (0, 1)$ be a real number. We are interested in the binary representation of $x$, namely \begin{equation*}
	x = 0.a_1 a_2 a_3 \dots
	\end{equation*}
	where the $a_n$ are either $0$ or $1$. Also, we require that the binary expansion never terminates in $1'$s. Then, the function $f : (0, 1) \rightarrow S$ such that $f(x)_n = a_n$ is easily seen to be 1-1, but it is not onto, since sequences that terminate in $1$'s will not be "reached" by the function. However, by the Schröder–Bernstein Theorem finding a 1-1 function from $g : S \rightarrow (0, 1)$ is enough for our purposes. To do this, let $g(A)_n = A_n$, where $g(A)_n$ represents the $n$th digit in the decimal expansion of a real number in the interval $(0, 1)$. $g$ is clearly 1-1, so we are done.
			    
	\exc{1.6.10}
		      	          
	\begin{enumerate}
		\item Let $F$ be the set of all functions from $\{0, 1\}$ to $\mathbb{N}$. Then, define $g : \mathbb{N}^2 \rightarrow F$ such that $g((a, b))$ is a function $f : \{0, 1\} \rightarrow \mathbb{N}^2$ such that \begin{equation*}
		      f(x) = \begin{cases}
		      a & x = 0 \\
		      b & x = 1
		\end{cases}
		\end{equation*}
		$g$ is easily seen to be 1-1 and onto, so $F \sim \mathbb{N}^2 \sim \mathbb{N}$, therefore $F$ is countable.
			      		      	        
		\item Let $F$ now be the set of all $f : \mathbb{N} \rightarrow \{0, 1\}$. Now let the function $g : F \rightarrow S$ be such that $g(f)_n = f(n)$ for every $n \in \mathbb{N}$ and every $f \in F$, where $S$ is the set of all sequences of $0'$s and $1'$s discussed in Exercise 1.6.4. Again, $g$ is easily seen to be a bijection, so $F \sim S \sim \mathbb{R}$, therefore $F$ is uncountable.
	\end{enumerate}
		      	          
	\exc{2.2.1}
	The sequence $f(n) = (-1)^n$ verconges to $0$ and $1$, but does not converge. This definition describes bounded sequences.
		      	          
	\exc{2.2.4}
		      	          
	\begin{enumerate}
		\item $f(n) = (-1)^n$.
		\item There is no such sequence. To see that, let $(a_n)$ be a sequence such that for every $N \in \mathbb{N}$ there is some $n \geq N$ such that $a_n = 1$ which also converges to some real number $L$. Now, assume $L \neq 1$. Since $(a_n)$ converges, there is some $M \in \mathbb{N}$ such that for all $m \geq M$ $|a_m - L| < |1-L|/2$, since $|1-L|/2 > 0$. By the construction of $(a_n)$, we can pick an $m \geq M$ such that $a_m = 1$. Then, we have $|1-L| < |1-L|/2$ which implies $1 < 1/2$, a contradiction. Therefore $(a_n)$ must converge to $1$.
		      	      	      	      	              
		\item $(0, 1, 0, 1, 1, 0, 1, 1, 1, 0, \dots)$.
	\end{enumerate}
		      	          
	\exc{2.2.5}
		      	              
	\begin{enumerate}
		\item We claim that $\lim a_n = 0$. Let $\epsilon > 0$ be arbitrary. Choose a natural number $N > 5$. Notice that whenever $n \geq N > 5$, $1 > 5/n \geq 0$, which means $0 = [[a_n]]$, therefore $|a_n - 0| = 0 < \epsilon$.
		      	      	      	      	                  
		\item We claim that $\lim a_n = 1$. Let $\epsilon > 0$ be arbitrary. Choose a natural number $N > 6$. Notice that whenever $n \geq N > 6$, $2 > (12+4n)/(3n) \geq 1$, which means $1 = [[a_n]]$, therefore $|a_n - 1| = 0 < \epsilon$.
	\end{enumerate}
		      	              
	\exc{2.2.6}
	Assume $a \neq b$. Then, there are naturals $N_1, N_2$ such that for every $n_1 \geq N_1$ and every $n_2 \geq N_2$, we have $|a_{n_1} - a| < |a-b|/2$ and $|a_{n_2} - b| < |a-b|/2$. By letting $N = \max (N_1, N_2)$, it is then true that for every $n \geq N$ $|a_n - a| < |a-b|/2$ and $|a_n - b| < |a-b|/2$. Adding both of these equations, we have $|a_n - a| + |a_n - b| < |a-b|$, which contradicts the triangle inequality, so we must have $a = b$.
		      	              
	\exc{2.2.7}
		      	              
	\begin{enumerate}
		\item The sequence $(-1)^n$ is frequently in $\{1\}$.
		\item Definition (i) is stronger, a sequence that is eventually in a set is also frequently in the set.
		      	      	      	      	                  
		\item A sequence $(a_n)$ converges to $a$ if, given any $\epsilon$-neighborhood $V_\epsilon(a)$ of $a$, the sequence is eventually in $V_\epsilon(a)$.
		      	      	      	      	                  
		\item The sequence $(1,2,1,2,1\dots)$ is not eventually in $(1.9, 2.1)$. However, any sequence with an infinite number of $2'$s is frequently in $(1.9, 2.1)$, since $2$ is in this set. 
	\end{enumerate}
		      	              
	\exc{2.2.8}
		      	              
	\begin{enumerate}
		\item Yes.
		\item Yes.
		\item The sequence $(0, 1, 0, 1, 1, 0, 1,1,1, 0 \dots)$ is a counterexample.
		\item A sequence is not zero-heavy if for all $M \in \mathbb{N}$ there exists $N \in \mathbb{N}$ such that for all $n$ satisfying $N \leq n \leq N + M$ we have $x_n \neq 0$.
	\end{enumerate}
		      	              
	\exc{2.3.1}
	\begin{enumerate}
		\item  Let $\epsilon > 0$ be arbitrary. Since $(x_n) \rightarrow 0$, there is an $N \in \mathbb{N}$ such that for all $n \geq N$, $x_n < \epsilon^2$. Then, $\sqrt{x_n} = |\sqrt{x_n} - 0| < \epsilon$, so $(\sqrt{x_n}) \rightarrow 0$.
		      	      	      	      	                  
		\item Since item (a) already proves the case where $x = 0$, we can assume $x > 0$. Now, let $\epsilon > 0$ be arbitrary. Since $(x_n) \rightarrow x$, there is an $N \in \mathbb{N}$ such that for all $n \geq N$ $|x_n-x| < \epsilon \sqrt{x}$. In that case, \begin{equation*}
		      |\sqrt{x_n}-\sqrt{x}| = |\sqrt{x_n}-\sqrt{x}| \frac{\sqrt{x_n}+ \sqrt{x}}{\sqrt{x_n}+ \sqrt{x}} = \frac{|x_n-x|}{\sqrt{x_n}+\sqrt{x}} \leq \frac{|x_n-x|}{\sqrt{x}} < \epsilon
		\end{equation*}
		and we are done.
	\end{enumerate}
		      	              
	\exc{2.3.2}
		      	              
	\begin{enumerate}
		\item Let $\epsilon > 0$ be arbitrary. Since $(x_n) \rightarrow 2$, we can choose a natural number $N$ such that $|x_n-2| < 3\epsilon/2$ for all $n \geq N$. Then, \begin{equation*}
		      \abs[\Big]{\frac{2x_n-1}{3}-1} = \frac{2}{3}|x_n-2| < \epsilon \, .
		\end{equation*}
			      		      	            
		\item Let $\epsilon > 0$ be arbitrary. Since $(x_n) \rightarrow 2$, we can choose a natural number $N_1$ such that $|x_n-2| < 2\epsilon$ for all $n \geq N_1$. We can also find a natural $N_2$ such that $\abs{2-x_n} < 1$, for all $n \geq N_2$, which implies $|x_n| > 1$. Let $N := \max(N_1, N_2)$. Then, for all $n \geq N$, we have \begin{equation*}
		      \abs[\Big]{\frac{1}{x_n}-\frac{1}{2}} = \abs[\Big]{\frac{2-x_n}{2x_n}} < \abs[\Big]{\frac{x_n-2}{2}} < \epsilon.
		\end{equation*}
	\end{enumerate}
		      	              
	\exc{2.3.3}
	Applying Theorem 2.3.4 twice, we have $l \leq \lim y_n \leq l$, which means $\lim y_n = l$.
		      	    
	\exc{2.3.4}
	\begin{enumerate}
		\item Applying the Algebraic Limit Theorem several times, we have: \begin{gather}
		      \nonumber \lim (\frac{1+2a_n}{1+3a_n-4a_{n}^2}) = \frac{\lim (1 + 2a_n)}{\lim (1 + 3a_n-4a_n^2)} = \\
		      \nonumber \frac{\lim (1) + 2\lim(a_n)}{\lim(1) + 3\lim(a_n) - 4 \lim(a_n) \lim(a_n)} = \frac{1}{1} = 1 \, .
		\end{gather}
			      		        
		\item \begin{equation*}
		      \frac{(a_n+2)^2-4}{a_n} = \frac{a_n (a_n+4)}{a_n} = a_n + 4
		\end{equation*}
		Then, \begin{equation*}
		\lim (\frac{(a_n+2)^2-4}{a_n}) = \lim (a_n) + \lim(4) = 4 \, .
		\end{equation*}
			      		        
		\item \begin{equation*}
		      \lim(\frac{\frac{2}{a_n}+3}{\frac{1}{a_n}+5}) = \lim (\frac{3a_n + 2}{5a_n + 1}) = 2 \, .
		\end{equation*}
	\end{enumerate}
		      
	\exc{2.3.5}
	Assume $(z_n) \rightarrow L$, for some real number $L$. We must show that both $(x_n)$ and $(y_n)$ are also convergent. Let $\epsilon > 0$ be arbitrary. There exists a natural number $N$ such that for all $n \geq N$ we have $\abs{z_n - L} < \epsilon$. Since $n \geq N \implies 2n-1 \geq N$, we also have $\abs{z_{2n-1} - L} < \epsilon$ for $n \geq N$. Similarly, $n \geq N \implies 2n \geq N$, therefore $\abs{z_{2n}-L} < \epsilon$. Therefore, for all $n \geq N$ we have both $\abs{x_n - L} < \epsilon$ and $\abs{y_n - L} < \epsilon$, since $z_{2n-1} = x_n$ and $z_{2n} = y_n$, so all three sequences converge to $L$.
	        
	For the converse, we assume $(x_n), (y_n) \rightarrow L$ for some real number $L$, and we must show $(z_n)$ also converges, in particular, we will show $(z_n) \rightarrow L$. Since $\abs{a} \geq 0$ for any real $a$, we have $\abs{y_n-L} = \abs{z_{2n} - L} \leq \abs{x_n-y_n} + \abs{y_n-L}$ for all natural $n$. Also, we can use the triangle inequality to see that $\abs{x_n-L} = \abs{z_{2n-1}- L} \leq \abs{x_n-y_n} + \abs{y_n - L}$. Now, let $\epsilon > 0$ be arbitrary. Choose a natural $N$ such that $\abs{x_n-y_n} < \epsilon/2$ and $\abs{y_n-L} < \epsilon/2$ for all $n \geq N$. Using the two inequalities just mentioned, we then have $\abs{z_{2n} - L} \leq \epsilon$ and $\abs{z_{2n-1} - L} < \epsilon$. This shows that $\abs{z_m - L} < \epsilon$ for all $m \geq 2N-1$, so $(z_n) \rightarrow 0$.
	        
	In the proof just given, we used the following fact: if $(x_n), (y, n) \rightarrow L$ for some real number $L$, then for every $\epsilon > 0$ there is some natural $N$ such that $\abs{x_n-y_n} < \epsilon$ for all $n \geq N$. To see that we can always do this, let $\epsilon > 0$ be arbitrary and use the fact that both the sequences converge to find $N_1, N_2 \in \mathbb{N}$ such that $\abs{x_n - L} < \epsilon/2$ for all $n \geq N_1$ and $\abs{y_m-L} < \epsilon/2$ for all $m \geq N_2$. Setting $N := \max(N_1, N_2)$ we have $\abs{x_n-L} < \epsilon/2$ $\abs{y_n-L} < \epsilon/2$ for all $n \geq N$. Summing the two inequalities, we get $\abs{x_n-L} + \abs{y_n-L} < \epsilon$, and we can use the triangle inequality to see that $\abs{x_n-y_n} \leq \abs{x_n-L} + \abs{y_n-L} < \epsilon$, as we wanted to show.
	        
	\exc{2.3.6}
	First, notice that
	\begin{equation*}
		\lim (1/n) = 0 \implies \lim(1+\sqrt{1+\frac{2}{n}}) = 2 \implies \lim (\frac{-2}{1+\sqrt{1+ \frac{2}{n}}}) = -1 \, .
	\end{equation*}
	Also, 
	\begin{equation*}
		b_n = n-\sqrt{n^2+2n} \cdot \frac{n+\sqrt{n^2+2n}}{n+\sqrt{n^2+2n}} = \frac{-2n}{n+\sqrt{n^2+2n}} = \frac{-2}{1+\sqrt{1+\frac{2}{n}}} ~ .
	\end{equation*}
	        
	Combining both results we have $\lim(b_n) = -1$.
	        
	\exc{2.3.7}
	\begin{enumerate}
		\item $x_n = n$ and $y_n = -n$.
		                  
		\item This is impossible. To see this, assume that $(x_n+y_n)$ and $(x_n)$ are convergent, while $(y_n)$ is not. By the Algebraic Limit Theorem, we have $\lim(y_n)=\lim((x_n+y_n)-x_n) = \lim(x_n+y_n)-\lim(x_n)$, so $(y_n)$ converges, a contradiction.
		                  
		\item $(1, 1/2, 1/3, 1/4 \dots)$.
		                  
		\item This is not possible. Assume for contradiction that $(a_n)$ is unbounded, $(b_n)$ is convergent and $(a_n-b_n)$ is bounded. By Theorem 2.3.2, there is a real number $M$ such that $M \geq \abs{b_n}$ for all $n$. By our initial assumption, there is also a real $L$ such that $L \geq \abs{a_n-b_n}$ for all $n$. Then, we have $L \geq \abs{a_n-b_n} \geq \abs{a_n} - \abs{b_n} \geq \abs{a_n} - M$, which means $L+M \geq \abs{a_n}$ for all $n$, which contradicts the assumption that $(a_n)$ was not bounded.
		                  
		\item $(a_n) = (0,0,0, \dots)$, $(b_n) = (1,2,3, \dots)$.
	\end{enumerate}
	        
	\exc{2.3.8}
	    
	\begin{enumerate}
		\item Assume $(x_n) \rightarrow x$. First, we use induction to show that \begin{equation*}
		      \lim (x_n^k) = x ^ k
		\end{equation*}
		for all natural $k$. The case $k = 1$ is trivial, so we assume the equality holds for $k$ and seek to show that it also holds for $k+1$. Applying the Algebraic Limit Theorem, we have $\lim(x_n^{k+1}) = \lim(x_n^k x_n) = x^k x = x^{k+1}$, as we wanted to show.
		            
		Now, let $p$ be a polynomial. We can write \begin{equation*}
		p(z) = \sum\limits_{i=0}^k a_i z^i
		\end{equation*}
		for every real $z$, some natural $k$ and a sequence of real numbers $(a_i)$. Then, we can use induction and the Algebraic Limit Theorem very similarly to the previous paragraph to see that \begin{equation*}
		\lim (p(x_n)) = \sum\limits_{i=0}^k a_i \lim(x_n^i) =\sum\limits_{i=0}^k a_i x^i = p(x)
		\end{equation*}
		therefore $p(x_n) \rightarrow p(x)$.
		            
		\item Let $(x_n)$ be the sequence where $x_n = 1/n$ for all natural $n$, and $f : {x_1, x_2, \dots} \rightarrow {0, 1}$ be such that \begin{equation*}
		      f(z) = \begin{cases}
		      0 & z \neq 0 \\
		      1 & z = 0
		\end{cases}
		\end{equation*}
		            
		Then, $\lim f(x_n) = \lim (0) = 0$ and $f(\lim x_n) = f(0) = 1$. Therefore, $\lim (f(x_n)) \neq f(\lim(x_n))$.
	\end{enumerate}
	        
	\exc{2.3.9}
	\begin{enumerate}
		\item Let $\epsilon > 0$ be arbitrary. Since $(a_n)$ is bounded, there is a real number $M \neq 0$ such that $M \geq \abs{a_n}$ for all natural $n$. Also, since $(b_n) \rightarrow 0$, there is a natural $N$ such that $\abs{b_n} \leq \epsilon/M$ for all $n \geq N$. Then, for all $n \geq N$ we have $\abs{a_n b_n} = \abs{a_n} \abs{b_n} \leq M \abs{b_n} < \epsilon$, so $(a_n b_n) \rightarrow 0$.
		      	        
		      We cannot use the Algebraic Limit Theorem to prove this since $(a_n)$ might not be convergent, even though it is bounded.
		      	        
		\item If $(b_n) \rightarrow b \neq 0$, then $(a_b b_n)$ converges $\iff (a_n)$ converges. The converse direction is a special case of the statement of the Algebraic Limit Theorem. In the other direction, notice that $a_n = (a_n b_n) / b_n$, so, $\lim ((a_n b_n) / b_n) = \lim(a_n b_n) / b = \lim (a_n)$, therefore $(a_n)$ converges.
		      	        
		\item Assume $\lim (a_n) = 0$ and $\lim (b_n) = b$. Since $(a_n)$ is convergent it is also bounded, therefore (a) guarantees that $\lim(a_b b_n) = 0 = \lim(a_n) \lim(b_n)$.
	\end{enumerate}
		    
	\exc{2.3.10}
	\begin{enumerate}
		\item $a_n = n$ and $b_n = n$ for all $n \in \mathbb{N}$ is a counterexample, since $\lim (a_n - b_n) = 0$ and neither $\lim(a_n)$ nor $\lim(b_n)$ exist.
		      	     
		\item Let $\epsilon > 0$ be arbitrary. Choose a natural number $N$ such that $\abs{b_n - b} < \epsilon$ for all $n \geq N$. Since $\abs{b_n} - \abs{b} \leq \abs{b_n - b}$ and $\abs{b} - \abs{b_n} \leq \abs{b_n - b}$, we have $\abs{\abs{b_n}-\abs{b}} \leq \abs{b_n-b} < \epsilon$ for all $n \geq N$ so $\abs{b_n} \rightarrow \abs{b}$.
		      	     
		\item By Theorem 2.3.3, $\lim ((b_n-a_n) + a_n) = \lim(b_n) = \lim (b_n - a_n) + \lim(a_n) = a$.
		      	     
		\item Let $\epsilon > 0$ be arbitrary. Choose an $N \in \mathbb{N}$ such that $\abs{a_n} < \epsilon$ for all $n \geq N$. Then, $0 \leq \abs{b_n-b} \leq a_n = \abs{a_n} < \epsilon$, so $\abs{b_n - b} < \epsilon$ for all $n \geq N$, therefore $(b_n) \rightarrow b$.
	\end{enumerate}
		 
	\exc{2.3.11}
	\begin{enumerate}
		\item Assume $(x_n) \rightarrow x$ and let $\epsilon > 0$ be arbitrary. Notice that \begin{equation*}
		      \abs{y_n-x} = \abs[\bigg]{\left (\sum\limits_{k=1}^{n} \frac{x_k}{n}\right)-x} = \abs[\Big]{\frac{1}{n} \sum\limits_{k=1}^{n} x_k-x} \leq \frac{1}{n} \sum\limits_{k=1}^n \abs{x_k-x}
		\end{equation*} for all natural $n$. Choose $N_1 \in \N$ such that $\abs{x_n-x} < \epsilon/4$ for all natural $n \geq N$. Then, we can write \begin{equation*}
		\abs{y_n-x} \leq \sum\limits_{k=1}^{N_1-1} \frac{\abs{x_k-x}}{n} + \sum\limits_{k=N_1}^{n} \frac{\abs{x_k-x}}{n} ~ .
		\end{equation*} Now, use the fact that the first term converges to $0$ to choose a natural number $N_2$ such that \begin{equation*}
		\sum\limits_{k=1}^{N_1-1} \frac{\abs{x_k-x}}{n} < \frac{\epsilon}{2}
		\end{equation*} for all $n \geq N_2$. By letting $N := \max (N_1, N_2)$, we can write \begin{equation*}
		\abs{y_n-x} \leq \frac{\epsilon}{2} + \sum\limits_{k=N_1}^{n} \frac{\abs{x_k-x}}{n} \leq \frac{\epsilon}{2} + \sum\limits_{k=N_1}^{n} \frac{\epsilon}{4n}
		\end{equation*} for all $n \geq N$. Notice that \begin{equation*}
		\sum\limits_{k=N_1}^{n} \frac{\epsilon}{4n} = \frac{n-N_1+1}{n} \cdot \frac{\epsilon}{4}
		\end{equation*} and, since $(n-N_1 + 1) / n < 2$ for all $n \geq N_1$, \begin{equation*}
		\sum\limits_{k=N_1}^{n} \frac{\epsilon}{4n} < 2 \cdot \frac{\epsilon}{4} = \frac{\epsilon}{2} ~.
		\end{equation*} Finally, \begin{equation*}
		\abs{y_n-x} \leq \frac{\epsilon}{2} + \frac{\epsilon}{2} = \epsilon
		\end{equation*} for all $n \geq N$, which means $(y_n) \rightarrow (x_n)$.
			     
		\item If for all naturals $n$ \begin{equation*}
		      x_n := \begin{cases}
		      0 & n \text{ is odd} \\
		      1 & n \text{ is even} \text{ ,}
		\end{cases}
		\end{equation*} then it is not hard to see that \begin{equation*}
		y_n = \begin{cases}
		\frac{n-1}{2n} & n \text{ is odd} \\
		\frac{1}{2} & n \text{ is even} ~ .
		\end{cases}
		\end{equation*} Therefore, $(y_n)$ is the "shuffled" sequence of $a_n = (n-1)/(2n)$ and $b_n = 1/2$, in the sense of Exercise 2.3.5. Notice that $\lim((n-1)/(2n)) = \lim(1/2 - 1/n) = 1/2 = \lim(a_n) = \lim(b_n)$, and by what was shown on Exercise 2.3.5 $(y_n)$ must converge, even though $(x_n)$ diverges.
	\end{enumerate}
		 
	\exc{2.3.12}
		 
	\begin{enumerate}
		\item True. For every $b \in B$ and every $n \in \N$ we have $a_n \geq B$, which implies $a \geq b$, by the Order Limit Theorem.
		      	     
		\item First, we show that every $a_n$ being in the complement of $(0, 1)$ implies the existence of some $N \in \N$ such that $a_n \geq 1$ for all $n \geq N$ or $a_n \leq 0$ for all $n \geq N$, as long as $a \neq 0$. Assume $a > 0$. Then, there is some $N \in \N$ such that $\abs{a-a_n} < a/2$. Now, assume for contradiction that there is some $m \geq N$ such that $a_m \leq 0$. Then, $\abs{a-a_m} = a-a_m < a/2$, which means $a_m > a/2 > 0$, a contradiction. For the case $a < 0$, choose $N \in \N$ such that $\abs{a_n-a} < 1-a$ for all $n \geq N$. Assume for contradiction that there is some $m \geq N$ such that $a_m \geq 1$. Then, $\abs{a_m-a} = a_m-a < 1-a$, which means $a_m < 1$, a contradiction. 
		      	     
		      If $a = 0$, then $a$ is already in the complement of $(0, 1)$, so assume $a \neq 0$. If $a > 0$, we have shown that there is some $n \geq N$ such that all $a_n \geq 1$, which, by a slightly modified version of the Order Limit Theorem, implies $a \geq 1$, so $a$ is in the complement of $(0, 1)$, and a similar argument follows when $a < 0$.
		      	     
		\item We have already shown that given any two real numbers, there is a rational number strictly between them. Therefore, we can make the sequence $(a_n)$ by choosing each $a_n$ such that $\sqrt{2} < a_n < \sqrt{2} + 1/n$ and $a_n \in \Q$. Every $a_n$ is rational by construction, but we claim $(a_n) \rightarrow \sqrt{2}$, which is irrational. To see this, let $\epsilon > 0$ be arbitrary and choose $N \in \N$ such that $N > 1/\epsilon$. Then, for every $n \geq N$, $a_n < \sqrt{2}+1/n < \sqrt{2}+\epsilon$, therefore $0 < a_n-\sqrt{2} = \abs{a_n-\sqrt{2}} < \epsilon$ for all $n \geq N$, so $(a_n) \rightarrow \sqrt{2}.$
	\end{enumerate} 
		 
	\item \begin{shortlemma}
	      Every Cauchy sequence is bounded.
	\end{shortlemma}
	\begin{proof}
		Let $(a_n)$ be a Cauchy sequence. Choose $N \in \N$ such that $\abs{a_n-a_m} < 1$ for all $n \geq N$. Then, $\abs{a_n} - \abs{a_N} \leq \abs{a_n-a_N} < 1$, therefore $\abs{a_n} < \abs{a_N} + 1$ for all $n \geq N$. Since every finite sequence is bounded, there is some real number $M_1$ such that $M_1 \geq \abs{a_n}$ for every $n < N$, so if we define $M := \max (M_1, \abs{a_N} + 1)$ we will have $M \geq \abs{a_n}$ for every natural $n$, therefore $(a_n)$ is bounded by $M$.  
	\end{proof}
	\item \begin{shorttheorem}
	      A sequence $(a_n)$ converges if and only if it is Cauchy.
	\end{shorttheorem}
		 
	\begin{proof}
		First, assume $(a_n) \rightarrow L$ for some $L \in \R$. Let $\epsilon > 0$ be arbitrary and choose $N \in \N$ such that $\abs{a_t - L} \leq \epsilon/2$ for all $t \geq N$. Then, for all $n,m \geq N$ we have $\abs{a_n - a_m} = \abs{a_n - L + L - a_m} \leq \abs{a_n-L} + \abs{a_m - L} < \epsilon/2 + \epsilon/2 = \epsilon$, so $(a_n)$ is Cauchy.
			     
		For the converse direction, assume $(a_n)$ is Cauchy. Now, let $\epsilon > 0$ be arbitrary and choose $N \in \N$ such that $\abs{a_n-a_m} < \epsilon/2$ for all $n,m \geq N$. Since every Cauchy sequence is bounded, we can define $s := \sup \set{a_n :n \in \N \text{ and } n \geq N}$. Since $s$ is a least upper bound, there is some $a_{n_0}$ with $n_0 \geq N$ such that $s-\epsilon/2 < a_{n_0}$, which implies $\abs{s-a_{n_0}} < \epsilon/2$. Then, $\abs{s-a_n} = \abs{s-a_{n_0} + a_{n_0}-a_n} \leq \abs{s-a_{n_0}} + \abs{a_{n_0}-a_n} < \epsilon/2 + \epsilon/2 = \epsilon$, so $(a_n) \ra s$.
	\end{proof}
\end{enumerate}


\end{document}
